% Options for packages loaded elsewhere
\PassOptionsToPackage{unicode}{hyperref}
\PassOptionsToPackage{hyphens}{url}
%
\documentclass[
]{article}
\usepackage{lmodern}
\usepackage{amssymb,amsmath}
\usepackage{ifxetex,ifluatex}
\ifnum 0\ifxetex 1\fi\ifluatex 1\fi=0 % if pdftex
  \usepackage[T1]{fontenc}
  \usepackage[utf8]{inputenc}
  \usepackage{textcomp} % provide euro and other symbols
\else % if luatex or xetex
  \usepackage{unicode-math}
  \defaultfontfeatures{Scale=MatchLowercase}
  \defaultfontfeatures[\rmfamily]{Ligatures=TeX,Scale=1}
\fi
% Use upquote if available, for straight quotes in verbatim environments
\IfFileExists{upquote.sty}{\usepackage{upquote}}{}
\IfFileExists{microtype.sty}{% use microtype if available
  \usepackage[]{microtype}
  \UseMicrotypeSet[protrusion]{basicmath} % disable protrusion for tt fonts
}{}
\makeatletter
\@ifundefined{KOMAClassName}{% if non-KOMA class
  \IfFileExists{parskip.sty}{%
    \usepackage{parskip}
  }{% else
    \setlength{\parindent}{0pt}
    \setlength{\parskip}{6pt plus 2pt minus 1pt}}
}{% if KOMA class
  \KOMAoptions{parskip=half}}
\makeatother
\usepackage{xcolor}
\IfFileExists{xurl.sty}{\usepackage{xurl}}{} % add URL line breaks if available
\IfFileExists{bookmark.sty}{\usepackage{bookmark}}{\usepackage{hyperref}}
\hypersetup{
  pdftitle={DATA607: HW2},
  pdfauthor={Neil Shah},
  hidelinks,
  pdfcreator={LaTeX via pandoc}}
\urlstyle{same} % disable monospaced font for URLs
\usepackage[margin=1in]{geometry}
\usepackage{graphicx,grffile}
\makeatletter
\def\maxwidth{\ifdim\Gin@nat@width>\linewidth\linewidth\else\Gin@nat@width\fi}
\def\maxheight{\ifdim\Gin@nat@height>\textheight\textheight\else\Gin@nat@height\fi}
\makeatother
% Scale images if necessary, so that they will not overflow the page
% margins by default, and it is still possible to overwrite the defaults
% using explicit options in \includegraphics[width, height, ...]{}
\setkeys{Gin}{width=\maxwidth,height=\maxheight,keepaspectratio}
% Set default figure placement to htbp
\makeatletter
\def\fps@figure{htbp}
\makeatother
\setlength{\emergencystretch}{3em} % prevent overfull lines
\providecommand{\tightlist}{%
  \setlength{\itemsep}{0pt}\setlength{\parskip}{0pt}}
\setcounter{secnumdepth}{-\maxdimen} % remove section numbering

\title{DATA607: HW2}
\author{Neil Shah}
\date{2/6/2020}

\begin{document}
\maketitle

\hypertarget{data-607-hw2-assignment-sql-and-r}{%
\subsubsection{DATA 607: HW2 Assignment SQL and
R}\label{data-607-hw2-assignment-sql-and-r}}

\hypertarget{neil-shah-02-06-2020}{%
\subsection{Neil Shah: 02-06-2020}\label{neil-shah-02-06-2020}}

\hypertarget{overview}{%
\subsection{Overview:}\label{overview}}

In this notebook we will be use relational databases (SQL) and R on
survey data. Specifically we will collect, wrangle and then analyze
survey data relating to 6 popular movies from a random group of
individuals. The overall goal here is to sucessfully pull survey data
into a SQL database and then into R, but see if we can uncover any
trends.

\hypertarget{experimental-methodology}{%
\subsection{Experimental Methodology}\label{experimental-methodology}}

I will perform the following steps in order

\begin{itemize}
\tightlist
\item
  Create a Survey
\item
  Distribute Survey
\item
  Collect Results into a SQL database
\item
  Collate SQL Database into R
\item
  Exploratory data analysis/wrangling
\item
  Analysis of Trends
\item
  Conclusions
\end{itemize}

\hypertarget{survey-design}{%
\subsection{Survey Design}\label{survey-design}}

I created a survey via Google Forms
\href{https://forms.gle/YthZSBA92WJ8YLai9}{available here} with the
following movies.

\begin{enumerate}
\def\labelenumi{\arabic{enumi}.}
\tightlist
\item
  The Irishman
\item
  The Joker
\item
  Parasite
\item
  Once Upon a Time in Hollywood
\item
  Roma
\item
  Into the Spiderverse.
\end{enumerate}

Movies were selected from a cursory search of top movies from 2018
onward via Google, strictly chosen by myself.

I proposed a linear scale for each movie from 0 to 5 as required
input/question for each participant. I specified that 1 would be defined
as ``least enjoyed'' and 5 ``most enjoyed'' with 3 being ``neutral''. I
included 0 as ``have not seen'' to ensure that all possibilities were
covered.

\hypertarget{distibution-of-survey}{%
\subsection{Distibution of Survey}\label{distibution-of-survey}}

I chose to create a link to said survey and disseminated through 5
WhatsApp groups that I am part of--the groups did not have overlapping
members and in sum the potential sample size was 36 unique participants.

I gave a 48 hour cut off time for responses--with the survey being
closed on Friday, February 6th 2020 at 12:00 AM EST.

\hypertarget{collect-results-into-sql-and-then-r}{%
\subsection{Collect Results into SQL and then
R}\label{collect-results-into-sql-and-then-r}}

After the survey closed, the results were exported as a .csv file and
imported into SQL.

\begin{verbatim}
CREATE TABLE data 
(id INT NOT NULL AUTO_INCREMENT, 
Irishman VARCHAR(255) NOT NULL, 
Joker VARCHAR(255) NOT NULL, 
Parasite VARCHAR(255) NOT NULL,
Hollywood VARCHAR(255) NOT NULL,
Roma VARCHAR(255) NOT NULL, 
Spiderman INT NOT NULL,
 PRIMARY KEY (id));
 
 LOAD DATA LOCAL INFILE 'C:\ProgramData\MySQL\MySQL Server 8.0\Uploads\moviedata.csv' 
 INTO TABLE data FIELDS TERMINATED BY ','
 LINES TERMINATED BY '\n' IGNORE 1 ROWS 
 (id, Irishman, Joker, Parasite, Hollywood, Roma, Spiderman);
\end{verbatim}

To interface between MySQL and R I used the package `RMySQL'

\begin{verbatim}
install.packages("RMySQL")
library(RMySQL)
Loading required package: DBI

>  con <- dbConnect(MySQL(),    user='root',    password='password',   dbname='movies', host='localhost')
> con
<MySQLConnection:0,0>
\end{verbatim}

Fetching a query

\begin{verbatim}

> moviequery <- dbSendQuery(con, "SELECT * FROM data")
> dbFetch(moviequery)
   id Irishman Joker Parasite Hollywood Roma Spiderman
1   1        0     2        5         3    0         5
2   2        0     0        0         4    0         4
3   3        4     4        0         0    0         5
4   4        3     0        0         0    1         4
5   5        0     0        0         0    0         0
6   6        0     0        0         0    0         0
7   7        2     5        0         0    4         4
8   8        3     3        0         4    0         5
9   9        0     0        5         0    5         0
10 10        0     0        0         2    1         5
11 11        0     4        5         2    0         3
12 12        1     5        0         0    0         4
13 13        0     5        0         0    0         3
14 14        0     5        5         5    0         0
15 15        0     0        0         0    0         5
16 16        0     4        0         0    0         5
17 17        0     0        0         0    0         4
18 18        5     0        5         5    0         5
\end{verbatim}

Another way to demonstrate that I can send queries

\begin{verbatim}
> dbFetch(dbSendQuery(con, "SELECT Joker FROM data"))
   Joker
1      2
2      0
3      4
4      0
5      0
6      0
7      5
8      3
9      0
10     0
11     4
12     5
13     5
14     5
15     0
16     4
17     0
18     0
\end{verbatim}

\hypertarget{initial-analysis-in-r}{%
\subsection{Initial analysis in R}\label{initial-analysis-in-r}}

To analyze the survey and to make my life easier, I'll load these
results into a dataframe.

\begin{verbatim}
> dbClearResult(dbListResults(con)[[1]])
[1] TRUE
> moviequery <- dbSendQuery(con, "SELECT * FROM data")
> df <- dbFetch(moviequery)
> df
   id Irishman Joker Parasite Hollywood Roma Spiderman
1   1        0     2        5         3    0         5
2   2        0     0        0         4    0         4
3   3        4     4        0         0    0         5
4   4        3     0        0         0    1         4
5   5        0     0        0         0    0         0
6   6        0     0        0         0    0         0
7   7        2     5        0         0    4         4
8   8        3     3        0         4    0         5
9   9        0     0        5         0    5         0
10 10        0     0        0         2    1         5
11 11        0     4        5         2    0         3
12 12        1     5        0         0    0         4
13 13        0     5        0         0    0         3
14 14        0     5        5         5    0         0
15 15        0     0        0         0    0         5
16 16        0     4        0         0    0         5
17 17        0     0        0         0    0         4
18 18        5     0        5         5    0         5
> dbListResults(con)
[[1]]
<MySQLResult:0,0,10>

> 
> dbDisconnect(con)
[1] TRUE
Warning message:
Closing open result sets
\end{verbatim}

Finally-for the fun analysis.

\begin{verbatim}
> head(df)
  id Irishman Joker Parasite Hollywood Roma Spiderman
1  1        0     2        5         3    0         5
2  2        0     0        0         4    0         4
3  3        4     4        0         0    0         5
4  4        3     0        0         0    1         4
5  5        0     0        0         0    0         0
6  6        0     0        0         0    0         0
> nrows(df)
Error in nrows(df) : could not find function "nrows"
> NROW(df)
[1] 18
\end{verbatim}

My survey had a total of 18 observations (survey results)--since I
already specified that 0 indicated an unseen movie {[}this was
suprisingl smart on my behalf{]}.

First I'll calculate the rating based on the sum of the rating values
divided by the non-zero value.

For example the Irishman:

\begin{verbatim}
> sum(df$Irishman)/ave(df$Irishman, FUN = function(x) sum(x!=0))[1]
[1] 3
\end{verbatim}

Also I thought it would interesting to tally the non-watch (or 0s) count
for each movie--Once again using the Irishman.

\begin{verbatim}
> (NROW(df$Irishman))-(ave(df$Irishman, FUN = function(x) sum(x!=0))[1])
[1] 12
\end{verbatim}

Dividing this result by the number of observations (18) would give us a
metric of non-watch percentage.

Collecting these all in a dataframe\ldots{}

\begin{verbatim}
> surveyratings <- c((sum(df$Irishman)/ave(df$Irishman, FUN = function(x) sum(x!=0))[1]),(sum(df$Joker/ave(df$Joker, FUN = function(x) sum(x!=0))[1])),(sum(df$Parasite)/ave(df$Parasite, FUN = function(x) sum(x!=0))[1]),(sum(df$Hollywood)/ave(df$Hollywood, FUN = function(x) sum(x!=0))[1]),(sum(df$Roma)/ave(df$Roma, FUN = function(x) sum(x!=0))[1]),(sum(df$Spiderman)/ave(df$Spiderman, FUN = function(x) sum(x!=0))[1]))
> surveyratings
[1] 3.000000 4.111111 5.000000 3.571429 2.750000 4.357143

> surveynonwatch <-c(((NROW(df$Irishman))-(ave(df$Irishman, FUN = function(x) sum(x!=0))[1])),((NROW(df$Joker))-(ave(df$Joker, FUN = function(x) sum(x!=0))[1])),((NROW(df$Parasite))-(ave(df$Parasite, FUN = function(x) sum(x!=0))[1])),((NROW(df$Hollywood))-(ave(df$Hollywood, FUN = function(x) sum(x!=0))[1])),((NROW(df$Roma))-(ave(df$Roma, FUN = function(x) sum(x!=0))[1])),((NROW(df$Spiderman))-(ave(df$Spiderman, FUN = function(x) sum(x!=0))[1])))
> surveynonwatch
[1] 12  9 13 11 14  4
> surveynonwatchpct <-(surveynonwatch/18)*100
> surveynonwatchpct 
[1] 66.66667 50.00000 72.22222 61.11111 77.77778 22.22222

> surveyDF <- data.frame('Movies'=titles,'SurveyRating'=surveyratings,'Nonwatch'=surveynonwatch,'NonWatch%'=surveynonwatchpct)
> surveyDF
     Movies SurveyRating Nonwatch NonWatch.
1  Irishman     3.000000       12  66.66667
2     Joker     4.111111        9  50.00000
3  Parasite     5.000000       13  72.22222
4 Hollywood     3.571429       11  61.11111
5      Roma     2.750000       14  77.77778
6 Spiderman     4.357143        4  22.22222
\end{verbatim}

Putting it all together in nice Data: Ink ratio form

\begin{verbatim}
> ggplot(surveyDF,aes(y=surveyDF$NonWatch.,x=surveyDF$Movies))+geom_bar(position="dodge", stat="identity") + labs(x='Movies',y='Unwatched %',title='n=18 Movie Survey Results: Non Watch Percent') + theme(panel.grid.major = element_blank(), panel.grid.minor = element_blank(),
+ panel.background = element_blank(), axis.line = element_line(colour = "black"))

> ggplot(surveyDF,aes(y=surveyDF$NonWatch,x=surveyDF$Movies))+geom_bar(position="dodge", stat="identity") + labs(x='Movies',y='Unwatched #',title='n=18 Movie Survey Results: Non Watch #') + theme(panel.grid.major = element_blank(), panel.grid.minor = element_blank(),
+ panel.background = element_blank(), axis.line = element_line(colour = "black"))
> ggplot(surveyDF,aes(y=surveyDF$SurveyRating,x=surveyDF$Movies))+geom_bar(position="dodge", stat="identity") + labs(x='Movies',y='Unwatched #',title='n=18 Movie Survey Results: Ratings') + theme(panel.grid.major = element_blank(), panel.grid.minor = element_blank(),
+ panel.background = element_blank(), axis.line = element_line(colour = "black"))
\end{verbatim}

\includegraphics{C:/Users/Neil/Documents/MSDS-Data Science/DATA 607/HW2surveyratings.png}
\includegraphics{C:/Users/Neil/Documents/MSDS-Data Science/DATA 607/HW2nonwatchpct.png}
\includegraphics{C:/Users/Neil/Documents/MSDS-Data Science/DATA 607/HW2nonwatch.png}

\hypertarget{imdb-comparision-analysis}{%
\subsection{IMDB Comparision Analysis}\label{imdb-comparision-analysis}}

\href{https://www.imdb.com/?ref_=nv_home}{IMDB} is a popular database
that contains granular data about movies, such as director, genre, cast
and ratings. I thought it would be interesting to use this database to
map metadata within my survey, and see if there were any other trends we
could discover.

I consulted the
\href{https://cran.r-project.org/web/packages/imdbapi/index.html}{API
documentation}, registered for an API key and then installed the
library.

\begin{verbatim}
library('imdbapi')
#Setting key to save me time
key =*****[intentionally blurred out by myself]
\end{verbatim}

The IMDBI can search it's database via Title or a unique ID. While I
could use the titles from my dataframe as an argument for the title
search, I felt it would be easier (it's only 6 movies) to hard code the
title IDs for easier search. I consulted IMDB url to generate my title
ID keys.

\begin{verbatim}
JokerID='tt7286456'
> RomaID='tt6155172'
> SpidermanID='tt4633694'
> HollywoodID='tt7131622'
> IrishmanID='tt1302006'
> ParasiteID='tt6751668'
\end{verbatim}

\hypertarget{building-imdb-database}{%
\subsubsection{Building IMDB Database}\label{building-imdb-database}}

Let's look at an example of an IMDB pull--I will store the metadata
within a dataframe for each movie.

\begin{verbatim}

RomaDF <- find_by_id(RomaID,api_key=key)
> head(RomaDF)
# A tibble: 2 x 25
  Title Year  Rated Released   Runtime Genre Director Writer Actors Plot  Language Country Awards Poster Ratings
  <chr> <chr> <chr> <date>     <chr>   <chr> <chr>    <chr>  <chr>  <chr> <chr>    <chr>   <chr>  <chr>  <list> 
1 Roma  2018  R     2018-11-21 135 min Drama Alfonso~ Alfon~ Yalit~ A ye~ Spanish~ Mexico  Won 3~ https~ <named~
2 Roma  2018  R     2018-11-21 135 min Drama Alfonso~ Alfon~ Yalit~ A ye~ Spanish~ Mexico  Won 3~ https~ <named~
# ... with 10 more variables: Metascore <chr>, imdbRating <dbl>, imdbVotes <dbl>, imdbID <chr>, Type <chr>,
#   DVD <date>, BoxOffice <chr>, Production <chr>, Website <chr>, Response <chr>
> 
\end{verbatim}

Repeating this for all the other movies

\begin{verbatim}
HollywoodDF <- find_by_id(HollywoodID,api_key=key)
> IrishDF <- find_by_id(IrishmanID,api_key=key)
> ParasiteDF <- find_by_id(ParasiteID,api_key=key)
> SpidermanDF <- find_by_id(SpidermanID,api_key=key)
\end{verbatim}

Awesome--now let's look dig into the granularity to see what data is
displayed.

\begin{verbatim}
> RomaDF$Awards
[1] "Won 3 Oscars. Another 238 wins & 198 nominations." "Won 3 Oscars. Another 238 wins & 198 nominations."
> RomaDF$Ratings
[[1]]
[[1]]$Source
[1] "Internet Movie Database"

[[1]]$Value
[1] "7.7/10"


[[2]]
[[2]]$Source
[1] "Metacritic"

[[2]]$Value
[1] "96/100"


> RomaDF$Genre
[1] "Drama" "Drama"
> RomaDF$Runtime
[1] "135 min" "135 min"
\end{verbatim}

As we can see, the IMDB database has a wealth of information! For this
project I am going to do a comparasion of my survey's ratings and IMDB;
specifically when looking at genre and runtime.

If we look at the IMDB rating

\begin{verbatim}
> RomaDF$Ratings[[1]][2]
$Value
[1] "7.7/10"
\end{verbatim}

It's based of a 1-10 point scale--while I used a 1-5 point. I'll
transform the IMDB rating to a comparative rating by dividing by 2.0.

TO do this I'll do the following:

\begin{enumerate}
\def\labelenumi{\arabic{enumi}.}
\tightlist
\item
  Pull the IMDB rating from the Ratings column 
\item
  Split the string across the `/'
\item
  Convert the string to a float
\item
  Divide by 2
\end{enumerate}

In action in one line of code

\begin{verbatim}
> RomaRating <- as.numeric(strsplit((RomaDF$Ratings[[1]][2]$Value),'/')[[1]][1])/2.0
> RomaRating
[1] 3.85
\end{verbatim}

Repeating the above for each movie and making a vector called ratings.

\begin{verbatim}
> ratings <- c((as.numeric(strsplit((IrishDF$Ratings[[1]][2]$Value),'/')[[1]][1])/2.0),(as.numeric(strsplit((JokerDF$Ratings[[1]][2]$Value),'/')[[1]][1])/2.0),(as.numeric(strsplit((ParasiteDF$Ratings[[1]][2]$Value),'/')[[1]][1])/2.0),(as.numeric(strsplit((HollywoodDF$Ratings[[1]][2]$Value),'/')[[1]][1])/2.0),(as.numeric(strsplit((RomaDF$Ratings[[1]][2]$Value),'/')[[1]][1])/2.0),(as.numeric(strsplit((SpidermanDF$Ratings[[1]][2]$Value),'/')[[1]][1])/2.0))
> ratings
[1] 4.00 4.30 4.30 3.90 3.85 4.20
\end{verbatim}

The previous method can be also be used to create vectors for
genre--note i'll only be taking the first genre definer.

\begin{verbatim}
> genre <- c((strsplit(IrishDF$Genre[1],',')[[1]][1]),(strsplit(JokerDF$Genre[1],',')[[1]][1]),(strsplit(ParasiteDF$Genre[1],',')[[1]][1]),(strsplit(HollywoodDF$Genre[1],',')[[1]][1]),(strsplit(RomaDF$Genre[1],',')[[1]][1]),(strsplit(SpidermanDF$Genre[1],',')[[1]][1]))
> 
> genre
[1] "Biography" "Crime"     "Comedy"    "Comedy"    "Drama"     "Animation"
\end{verbatim}

Finally the same for runtimes

\begin{verbatim}

> runtimes <- c((strsplit(IrishDF$Runtime[1],' ')[[1]][1]),(strsplit(JokerDF$Runtime[1],' ')[[1]][1]),(strsplit(ParasiteDF$Runtime[1],' ')[[1]][1]),(strsplit(HollywoodDF$Runtime[1],' ')[[1]][1]),(strsplit(RomaDF$Runtime[1],' ')[[1]][1]),(strsplit(SpidermanDF$Runtime[1],' ')[[1]][1]))
> runtimes
[1] "209" "122" "132" "161" "135" "117"
\end{verbatim}

Great--now combining these all in one IMDB dataframe.

\begin{verbatim}
> IMDBDF <-data.frame('Movies'=titles,'Run Time'=runtimes,'IMDBRatings'=ratings,'Genre'=genre)
> IMDBDF
     Movies Run.Time IMDBRatings     Genre
1  Irishman      209        4.00 Biography
2     Joker      122        4.30     Crime
3  Parasite      132        4.30    Comedy
4 Hollywood      161        3.90    Comedy
5      Roma      135        3.85     Drama
6 Spiderman      117        4.20 Animation
\end{verbatim}

\hypertarget{conclusions}{%
\subsection{Conclusions}\label{conclusions}}

\hypertarget{recommendations}{%
\subsubsection{Recommendations}\label{recommendations}}

\begin{enumerate}
\def\labelenumi{\arabic{enumi}.}
\item
  \textbf{Unique identifers:} To elminate the possibility of
  multiple-sampling, a unique identifier (email address) could be used
  for each survey participant. While this might increase the
  non-response rate (individauls who want to remain anonymous), it also
  elimninates the possibility of multiple entries.
\item
  \textbf{Random selection of movies:} I chose the movies randomly but
  there can be internal biases--perhaps I chose movies that only I saw
  subconciously. I propose next time that a list of popular movies
  (metric could be rating, boxoffice values or weeks on top list),
  inserted into an array and then 6 are randomly chosen to sample. This
  could eliminate any selection bias .
\item
  \textbf{Random selection of genres/medium:} Similar to \#2--perhaps
  using top movies from each genre would be a better way to gain
  diverisity/sample said movies. In addition including Netflix, Amazon
  Prime and other media would be interesting as well.
\item
  \textbf{More metrics on movie user:} Additional of questions on
  frequency of movies watched or favorite genre could reveal more themes
  or trends.
\item
  \textbf{Alternative Samplng:} Since I sent the survey through my own
  WhatsApp group, only friends and families saw said survey. If
  resources were available an internet poll or through CUNY could yield
  a large and diverse sample.
\end{enumerate}

\end{document}
